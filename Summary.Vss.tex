%Pakete;
%A4, Report, 12pt
\documentclass[ngerman,a4paper,12pt]{scrreprt}
\usepackage[a4paper, right=20mm, left=20mm,top=30mm, bottom=30mm, marginparsep=5mm, marginparwidth=5mm, headheight=7mm, headsep=15mm,footskip=15mm]{geometry}

%Papierausrichtungen
\usepackage{pdflscape}
\usepackage{lscape}

%Deutsche Umlaute, Schriftart, Deutsche Bezeichnungen
\usepackage[utf8]{inputenc}
\usepackage[T1]{fontenc}
\usepackage[ngerman]{babel}

%quellcode
\usepackage{listings}

%tabellen
\usepackage{tabularx}

%listen und aufzählungen
\usepackage{paralist}

%farben
\usepackage[svgnames,table,hyperref]{xcolor}

%symbole
\usepackage{latexsym,textcomp}
\usepackage{amssymb}

%font
\usepackage{helvet}
\renewcommand{\familydefault}{\sfdefault}

%durch- und unterstreichen
\usepackage{ulem}

%Abkürzungsverzeichnisse
\usepackage[printonlyused]{acronym}

%Bilder
\usepackage{graphicx} %Bilder
\usepackage{float}	  %"Floating" Objects, Bilder, Tabellen...
\usepackage[space]{grffile} %Leerzechen Problem bei includegraphics
\usepackage{wallpaper} %Seitenhintergrund setzen
\usepackage{transparent} %Transparenz

%Tikz, Mindmaps, Trees
\usepackage{tikz}
\usetikzlibrary{mindmap,trees}
\usepackage{verbatim}

%for
\usepackage{forloop}
\usepackage{ifthen}

%Dokumenteigenschaften
\title{Summary Vss}
\author{Tobias Blaser}
\date{\today{}, Uster}


%Kopf- /Fusszeile
\usepackage{fancyhdr}
\usepackage{lastpage}

\pagestyle{fancy}
	\fancyhf{} %alle Kopf- und Fußzeilenfelder bereinigen
	\renewcommand{\headrulewidth}{0pt} %obere Trennlinie
	\fancyfoot[L]{\jobname} %Fusszeile links
	\fancyfoot[C]{Seite \thepage/\pageref{LastPage}} %Fusszeile mitte
	\fancyfoot[R]{\today{}} %Fusszeile rechts
	\renewcommand{\footrulewidth}{0.4pt} %untere Trennlinie

%Kopf-/ Fusszeile auf chapter page
\fancypagestyle{plain} {
	\fancyhf{} %alle Kopf- und Fußzeilenfelder bereinigen
	\renewcommand{\headrulewidth}{0pt} %obere Trennlinie
	\fancyfoot[L]{\jobname} %Fusszeile links
	\fancyfoot[C]{Seite \thepage/\pageref{LastPage}} %Fusszeile mitte
	\fancyfoot[R]{\today{}} %Fusszeile rechts
	\renewcommand{\footrulewidth}{0.4pt} %untere Trennlinie
}

\usepackage{changepage}

% Abkürzungen für Kapitel, Titel und Listen
\input{commands/shortcutsListAndChapter}
\input{commands/TextStructuringBoxes}

%links, verlinktes Inhaltsverzeichnis, PDF Inhaltsverzeichnis
\usepackage[bookmarks=true,
bookmarksopen=true,
bookmarksnumbered=true,
breaklinks=true,
colorlinks=true,
linkcolor=black,
anchorcolor=black,
citecolor=black,
filecolor=black,
menucolor=black,
pagecolor=black,
urlcolor=black
]{hyperref} % Paket muss unbedingt als letzes eingebunden werden!

\usepackage{graphicx}
\begin{document}

% Inhaltsverzeichnis
\tableofcontents
\clearpage

\ch{Verteilte Systeme}

\definition{Verteilte Systeme}{Zusammenarbeit von Komponenten auf vernetzten Rechnern, die sich durch Nachrichtenaustausch koordinieren.}

Konsequenzen:
\ul
	\li Komponenten können nebenläufig arbeiten.
	\li Komponenten können unabhängig voneinander ausfallen.
	\li Es gibt keine globale Uhr. 
	\li Es ist schwierig einen globalen Zustand zu definieren.
\ulE

\expl{Logische Clock}{Threads, die sich synchronisieren, keine reale Uhr}
Keine Globale Uhr:
\ul
	\li Aktionen / Events, die zu einem bestimmten Zeitpunkt ausgelöst werden, müssen koordiniert werden.
	\li Zwei Rechner in einem Netz besitzen fast nie exakt die selbe Zeit.
	\li Wie könnte man dies überhaupt prüfen? Die Signalzeiten sind ja nur statistisch bekannt.
\ulE

\se{Herausforderungen verteilter Systeme}
\ul
	\li Nebenläufigkeit / Concurrency
	\li Fehlerverarbeitung / Fehlertoleranz
	\li Sicherheit
	\li Offenheit
	\li Heterogenität
	\li Skalierbarkeit
	\li Transparenz (Verbergen der Komplexität)
\ulE

\se{CAP Theroem}
\definition{CAP-Theorem}{ein verteiltes System kann zwei der folgenden
Eigenschaften gleichzeitig erfüllen, jedoch nicht alle drei.}

\ul
	\li Konsistenz (C): Alle Knoten sehen zur selben Zeit dieselben Daten. Diese
Konsistenz sollte nicht verwechselt werden mit der Konsistenz aus der
ACID-Transaktionen, die nur die innere Konsistenz eines Datenbestandes
betrifft.
	\li Verfügbarkeit (A): Alle Anfragen an das System werden stets beantwortet.
	\li Partitionstoleranz (P): Das System arbeitet auch bei Verlust von
Nachrichten, einzelner Netzknoten oder Partition des Netzes weiter.
	\li Da nur zwei dieser drei Anforderungen in verteilten Systemen gleichzeitig
vollständig erfüllt sein können, wird das CAP-Theorem oft als Dreieck
visualisiert, bei dem eine konkrete Anwendung sich auf eine der Kanten
einordnen lässt.Die System-Eigenschaften C, A und P können dabei als
graduelle Grössen gesehen werden.

\ulE

\se{Offenheit}
Verteilte Systeme müssen offen sein
\ul
	\li Notwendig für Anpassungen und Erweiterungen an neue Anforderungen
	\li Mittel zu offenen Systemen
		\ul
			\li Schnittstelln
			\li EInheitliche Kommunkationsmechanismen
		\ulE
\ulE

\se{Sicherheit}
\ul
	\li Vertraulichkeit:
 \ra Daten können nur von dem
gewünschten Empfänger gelesen werden.
\li Integrität: \ra
Die Daten wurden während der
Übertragung nicht verändert.
\li Authentizität: \ra
Die Daten wurden tatsächlich
von der Person gesendet, die behauptet, der Sender zu sein.
\li Verfügbarkeit: \ra
Ein Dienst darf durch eine (Denial of Service) Attacke nicht außer Kraft
gesetzt werden.
\li Sicherheit für mobilen Code: \ra
Mobiler Code darf die lokale Ressource nicht beschädigen und
umgekehrt.

\ulE

\se{Skalierbarkeit}
\expl{Skalierbarkeit}{Algorithmen, Protokolle und Prozeduren, die mit einigen wenigen
Systemkomponenten gut funktionieren (effektiv und effizient), sollen
auch mit vielen Komponenten gut funktionieren, skalierbar sein.
}

\se{Fehlertoleranz}

\se{Nebenläufigkeit}

\se{Transparenz}


\ch{Systemmodelle}
\exam{Systemmodell relevant für Prüfung}

\expl{Architekturmodell}{beschreibt Beziehungen und Verteilung von Komponenten in einem verteilten Syste}

\se{Systemarchitekturen}
\sse{Client-Server}
\img{img/v1.1.jpg}{}{0.75}{}

\sse{Mehrfache Server}
\img{img/v1.2.jpg}{}{0.75}{}

\sse{Proxy-Server und Cache}
\img{img/v1.3.jpg}{}{0.75}{}

\sse{Peer-to-Peer}
\img{img/v1.4.jpg}{}{0.75}{}

\sse{Appets}
\img{img/v1.8.jpg}{}{0.75}{}

\sse{MVC als Architektur Pattern}
Präsentation läuft auf dem Client, Model auf dem Server, Kontrolle über das Netzwerk.

\sse{Three Tier Modell}
\img{img/v1.9.jpg}{}{0.75}{}


\se{Ubiquitäre Systeme}

Mobile Computing: Bewegung ist integraler Bestandteil des
täglichen Lebens. 

\sse{Probleme}
\ul
	\li Schwankungen in Netzqualität
	\li vermindertes Vertrauen und niedrige Robustheit mobiler Endgeräte
	\li Einschränkungen durch Gewicht, Grösse und Batteriekapazität
\ulE

\sse{Grundlagen}
\ul
	\li Mobile Networking mit Mobile IP, AdHoc-Protokollen, ...
	\li Mobile Information Access,...
	\li Energiespartechniken (Speichermanagement, Prozessorscheduling)
	\li Ortsabhängigkeit (GPS, Lokalisierung, Systemverhalten,...)
\ulE

\ul
	\li  Unsichtbarkeit:
	\ra Computertechnologie soll vollständig aus dem Bewusstsein der Nutzer
	verschwinden
	\li Lokale Skalierbarkeit:
	\ra Anstieg der Bandbreite, Energie eines Nutzers und Anstieg der Nutzer
	in einem „Dienstraum“. Lokalität ändert sich rasch und es gibt mehr
	lokale als entfernte Interaktionspartner.
	\li Ausgleich:
	\ra Unterschiedliche Ausstattung unterschiedlicher Diensträume werden
	vom System "homogenisiert.
	\li Information Appliance:
	\ra Rückverlagerung der Funktionalität von Rechnern in die Anwendung.
\ulE

\se{Grundlegende Modelle}

\se{Interaktionsmodell}
\ul
	\li Timing am schwierigsten
	\li Uhren gelten nur lokal
	\li eingeschränkte Kommunikationskanäle (Verzögerungen senden/empfangen, Bandbreitenschwankungen, Jitter)
\ulE

\se{Fehlermodell}
\img{img/v1.5.jpg}{}{0.75}{}

\se{Sicherheitsmodell}
\img{img/v1.6.jpg}{}{0.75}{}
\ul
	\li Prozess Bedrohung (Serverprozess kennt Client zu wenig genau \ra Berechtigungen ungewiss, Client unsicher, ob Antwort wirklich vom Server)
\ulE

\se{RM-ODP Architekturansatz}
Metamodell für offene verteilte Systeme
\img{img/v1.7.jpg}{}{0.75}{}

Systembetrachtung setzt sich aus View Points zusammen (Gesichtswinkeln), weil  sonst zu komplex.

\img{img/v1.10.jpg}{}{0.75}{}
\img{img/v1.11.jpg}{}{0.75}{}

\exam{Beispiel mit File Fetch Java mitnehmen. Beispiel mit Socket kommt garantiert an Prüfung.}






\end{document}
