%Pakete;
%A4, Report, 12pt
\documentclass[ngerman,a4paper,12pt]{scrreprt}
\usepackage[a4paper, right=20mm, left=20mm,top=20mm, bottom=30mm, marginparsep=5mm, marginparwidth=5mm, headheight=7mm, headsep=15mm,footskip=15mm]{geometry}

%Papierausrichtungen
\usepackage{pdflscape}
\usepackage{lscape}

%Deutsche Umlaute, Schriftart, Deutsche Bezeichnungen
\usepackage[utf8]{inputenc}
\usepackage[T1]{fontenc}
\usepackage[ngerman]{babel}

%quellcode
\usepackage{listings}

%tabellen
\usepackage{tabularx}

%listen und aufzählungen
\usepackage{paralist}

%farben
\usepackage[svgnames,table,hyperref]{xcolor}

%symbole
\usepackage{latexsym,textcomp}

%font
\usepackage{helvet}
\renewcommand{\familydefault}{\sfdefault}

%Abkürzungsverzeichnisse
\usepackage[printonlyused]{acronym}

%Bilder
\usepackage{graphicx} %Bilder
\usepackage{float}	  %"Floating" Objects, Bilder, Tabellen...
\usepackage[space]{grffile} %Leerzechen Problem bei includegraphics
\usepackage{wallpaper} %Seitenhintergrund setzen
\usepackage{transparent} %Transparenz

%for
\usepackage{forloop}
\usepackage{ifthen}

%Dokumenteigenschaften
\title{Repetitionsfragen Vss}
\author{Tobias Blaser}
\date{\today{}, Rapperswil}


%Kopf- /Fusszeile
\usepackage{fancyhdr}
\usepackage{lastpage}

\pagestyle{fancy}
	\fancyhf{} %alle Kopf- und Fußzeilenfelder bereinigen
	\renewcommand{\headrulewidth}{0pt} %obere Trennlinie
	\fancyfoot[L]{Seite \thepage/\pageref{LastPage}} %Fusszeile mitte
	\fancyfoot[R]{\today{}} %Fusszeile rechts
	\renewcommand{\footrulewidth}{0.4pt} %untere Trennlinie

%Kopf-/ Fusszeile auf chapter page
\fancypagestyle{plain} {
	\fancyhf{} %alle Kopf- und Fußzeilenfelder bereinigen
	\renewcommand{\headrulewidth}{0pt} %obere Trennlinie
	\fancyfoot[L]{Seite \thepage/\pageref{LastPage}} %Fusszeile mitte
	\fancyfoot[R]{\today{}} %Fusszeile rechts
	\renewcommand{\footrulewidth}{0.4pt} %untere Trennlinie
}

\usepackage{changepage}

% Abkürzungen für Kapitel, Titel und Listen
\input{commands/shortcutsListAndChapter}
\input{commands/TextStructuringBoxes}

%links, verlinktes Inhaltsverzeichnis, PDF Inhaltsverzeichnis
\usepackage[bookmarks=true,
bookmarksopen=true,
bookmarksnumbered=true,
breaklinks=true,
colorlinks=true,
linkcolor=black,
anchorcolor=black,
citecolor=black,
filecolor=black,
menucolor=black,
pagecolor=black,
urlcolor=black
]{hyperref} % Paket muss unbedingt als letzes eingebunden werden!

\usepackage{graphicx}
\begin{document}

% Inhaltsverzeichnis
\tableofcontents

\vspace{1cm}
\section*{Antworten zu den Repetitionsfragen}
Falls vorhanden befinden sich diese im GitHub Repository. Ergänzungen oder ganze Antwortensets sind jederzeit herzlich willkommen.

\noindent\url{https://github.com/moonline}

\clearpage

\ch{Systemmodelle}
\ol
	\li Was ist ein Architekturmodell?
	\li Erklären Sie die folgenden Systemarchitekturen:
		\ol
			\li Client/Server Modell
			\li Proxy-Server und Cache
			\li Peer-to-Peer
			\li Applets
			\li Thin Client / Fat Server
			\li MVC Pattern
			\li Three Tier
			\li ISO OSI Layer Modell
		\olE
	\li Was sind Ubiquitäre Systeme?
	\li Was ist das Interaktionsmodell?
	\li Was ist das Fehlermodell?
	\li Was ist das Sicherheitsmodell?
	\li Erklären Sie den RM-ODP Architekturansatz.
\olS


\ch{Interprozesskommunikation}
\olR
	\li Stellen Sie synchroner asynchroner Kommunikation gegenüber.
	\li Was sind Sockets und Ports?
	\li Welche Probleme können bei IP, TCP oder UDP auftreten? Wie stark garantiert ist die Übertragung?
	\li Erklären Sie das Fehlermodell von UDP.
	\li Wann setzen Sie UDP ein?
	\li Programmieren Sie einen einfachen UDP Sender und Empfänger. Machen Sie selbiges für TCP.
	\li Wie wirken sich Blockierungen und Timeouts auf die send() und receive() Methoden von UDP in Java aus?
	\li Erklären Sie das Fehlermodell von TCP.
	\li Erklären Sie, was Gruppenkommunikation ist, und wie sie funktioniert. Erklären Sie IP Multicast.
	\li Erklären Sie, externes und internes Marshalling.
	\li Stelle Sie direkter Socketnutzung der Nutzung durch Middleware gegenüber.
	\li Was passiert bei der Objektserialisierung in Java? Wozu können Sie sie nutzen? Wo liegen die Fallstricke? Welche Nachteile hat sie?
	\li Bauen Sie ein kleines Beispielprogramm, in dem Sie ein Objekt Serialisieren, über einen UDP Socket versenden, empfangen und das Objekt wiederherstellen und Attribute ausgeben.
	\li Wie markieren Sie nicht zu serialisierende Objekte? Wie können Sie die Serialisierungsprozedur anpassen?
	\li Wozu dient die SerialVerUID? Wo kann es damit Probleme geben?
	\li Was können Sie mit Java2XML machen?
	\li Was müssen Sie bei der Referenzierung von remote Objekten in einem Verteilten System beachten?
	\li Wozu dienen die NIO-Channels in Java?
\olS


\ch{Verteilte Objekte und entfernte Aufrufe}
\olR
	\li Warum sollte der Benutzer erkennen können, ob es sich bei einem Methodenaufruf um einen lokalen oder entfernten Aufruf handelt?
	\li Welche Dienste sollte die Middleware zur Verfügung stellen?
	\li Erklären Sie das RRA Protokoll schematisch. Welche Probleme führt es mit sich?
	\li Zeigen Sie das Fehlermodell der RRA Protokolls auf. (Server und Clientseitig)
	\li Erklären Se die vier Arten von Aufrufsemantik.
	\li Welche Idee steckt hinter 'Remote Procedure call'? Welche Aufrufarten machen Sinn, welche nicht? (Call by Value, Call by Reference, Zugriff auf lokale Variablen des entfernten Systems)?
	\li Was ist eine IDL? Wie ist die CROBA IDL aufgebaut?
	\li Was ist RPC und wie funktioniert es?
	\li Was ist der Unterschied zwischen einer Methode und einer Procedure?
	\li Was ist RMI? Wie funktioniert es?
\olS

\se{RMI}
\olR
	\li Skizzieren SIe konzepthaft, was sie alles unternehmen müssen, um einen remote Methodenaufruf mit RMI realisieren zu können.
	\li Abb. \ref{rmiprox}: Erklären Sie die Grafik.
		\img{img/v4.3.jpg}{}{0.75}{rmiprox}
	\li Welche Rolle übernehmen der Proxy und welche der Skeletton?
	\li Welche Aufgabe übernimmt der Dispatcher?
	\li Wie realisieren Sie bei RMI eine Objektübergabe by Value? Wie können Sie Ansatzweise 'Object by Reference' realisieren?
	\li Erklären Sie, was remote Garbage Collection ist.
	\li Wozu dient das dynamische Laden von Klassen? Wie realisieren Sie es?
	\li Erklären Sie, wie sie einen RMI Server schlafen legen können und nur dann aufwecken, wenn wirklich etwas ansteht.
\olS


\ch{Messaging}
\olp{
	\li Erklären Sie die zwei Hauptvorteile von Messaging gegenüber RMI.
	\li Was ist indirekte Kommunikation?
	\li Erklären Sie jede Variante von räumlich/zeitlich gekoppelten Systemen und zeigen Sie, wo sie jeweils eingesetzt werden. \\
	\begin{tabular}{|l|l|l|}
		\hline
		& räumlich gekoppelt & räumlich nicht gekoppelt \\
		\hline
		zeitlich gekoppelt & & \\
		\hline
		zeitlich nicht gekoppelt & & \\
		\hline
	\end{tabular}
}

\se{Gruppenkommunikation}
\olp{
	\li Nennen Sie einige Anwendungsfälle für Gruppenkommunikation
	\li Wo liegt der Unterscheid zwischen offenen und geschlossenen sowie überlappenden und nicht überlappenden Gruppen?
	\li Welche Anforderungen punkto Zuverlässigkeit und Ordnung werden bei P2P Verbindungen und bei Multicast an die Kommunikation gestellt? Erklären Sie die drei Arten von Ordnungen, die es gibt.
	\li Was ist JGroups? Was sind JGroups Channels und JGroups Building Blocks?
}
\se{Publisher / Subscriber}
\olp{
	\li Erklären Sie das Prinzip von Publisher/Subscriber. Was sind Rollen/filter in diesem Zusammenhang?
	\li Was sind die Varianten Topic- und was Typen basierte P/S Systeme?
	\li Welche Komponenten stellt sicher, dass alle Teilnehmer die Nachricht erhalten haben, bevor sie entsorgt wird?
	\li Was ist ein Broker?
	\li Skizzieren Sie die typische Implementationsarchitektur von P/S.
}
\se{Message Queues}
\olp{
	\li Was sind Message Queues? Wo ist der Unterschied zu P/S? Welche drei Empfangsarten gibt es bei Message Queues?
	\li Was passiert bei Message Queues, wenn mehrere Consumer Nachrichten abholen?
	\li In den Übungen wurde für die Übertragung der Nachrichten zwischen dem Producer/Consumer und der Queue eine Session aufgebaut. Macht dies Sinn? Warum?
	\li Skizzieren Sie das Programmiermodell der Message Queues.
	\li Was ist JMS?
	\li Was ist MOM?
	\li Was sind virtuell vollverknüpfte Netze?
}

\ch{Verteilte Dateisysteme}
\olp{
	\li Nennen Sie einige Anforderungen an ein verteiltes Dateisystem
	\li Was ist der Hauptzwecke eines verteilten Dateisystems
	\li Welche Probleme müssen gelöst werden bei verteilten Dateisystemen?
	\li Warum müssen Operationen wiederholbar sei?
}
\se{NFS}
\olp{
	\li Erklären Sie schematisch, wie NFS funktioniert.
	\li Was ist Dateidelegation?
	\li Auf welche Art von Dateien ist NFS ausgerichtet?
}

\se{AFS}
\olp{
	\li Erklären Sie schematisch, wie AFS funktioniert.
	\li Auf welche Art von Dateien ist AFS ausgerichtet?
	\li Wo liegt der Hauptunterschied von NFS und AFS?
}

\se{HDFS}
\olp{
	\li Nennen Sie einige wichtige Anforderungen, die zur Entwicklung von Hadoop Distributed Filesystem geführt haben
	\li Welche Eigenschaften besitzt HDFS?
	\li Wodurch unterscheidet sich HDFS von den andern Systemen?
	\li Nennen Sie die Anforderungen an Ausfallsicherheit, Daten Recoverabilität, Komponenten Recovery, Konsistenz und Skalierbarkeit, die an Hadoop gestellt werden.
	\li Wie senkt Hadoop die menge an Transfers? Warum arbeiten Tasks mit so wenigen Blöcken wie mögich?
	\li Was passiert mit einem Node nach einem Ausfall?
	\li Was ist das Map Reduce Framework? Wozu dient es? Erklären Sie es am Beispiel ``WordCount''.
	\li Skizzieren Sie die Systemarchitektur bei Hadoop.
	\li Wie laufen Datenzugriffe ab?
	\li Wie ist eine Hadoop Cluster Infrastructure aufgebaut?
}


\ch{Namesdienste}
\olp{
	\li Was ist ein Namensdienst? Wozu wird er benutzt
	\li Was ist ein Context? Was ein Subcontext?
	\li Was bedeutet es, einen Context ``zu binden''?
	\li Was ist ein Namensraum (Namespace)?
	\li Was ist ein Naming System?
	\li Was ist ein Verzeichnisdienst?
	\li Nennen Sie einige Anforderungen an Verzeichnisdienste.
	\li Was sind Aliases?
	\li Erklären Sie den Unterschied zwischen iterativer und rekursiver Namensauflösung am Beispiel DNS, und NFS. Was ist Serverkontrollierte AUflösung?
	\li Was sind Directory udn Discovery Services?
	\li Wie funktioniert GNS?
	\li Wie funktioniert X.500?
}

\se{LDAP}
\olp{
	\li Was ist LDAP und Wie funktioniert es?
	\li Stellen Sie LDAP OSI DAP gegenüber.
	\li Wie sind LDAP Verzeichnisse aufgebaut?
	\li Wie ist ein LDAP Eintrag aufgebaut?
	\li Was ist eine LDAP Objektklasse?
	\li Erklären Sie das LDAP Funktionsmodell.
	\li Erklären Sie das LDAP Namensmodell.
	\li Erklären Sie das LDAP Kommunikationsmodell.
	\li Erklären Sie das LDAP Konfigurationsmodell.
	\li Erklären Sie das logische LDAP Modell.
}

\se{JNDI}
\olp{
	\li Was ist JNDI?
	\li Wie ist JNDI aufgebaut?
}


\ch{P2P}
\olp{
	\li Welche Vorteile bietet P2P gegenüber Client/Server? Wo gibt es Nachteile?
	\li Was passiert in einem P2P System beim Ausfall eines Knotens?
	\li Welche Aufgaben muss eine P2P Middleware übernehmen?
	\li Was sind Routing Overlays?
	\li Wie werden Knoten identifiziert?
	}
\se{Routing}
\olp{
	\li Beschreiben Sie das ``Routing Problem'' bei P2P Netzwerken.
	\li Erklären Sie, wie zirkuläres Routing funktioniert.
	\li Erklären Sie, wie das Pastry Routing funktioniert.
	\li Wie ist der Pastry Note State aufgebaut?
	\li Was ist Lokalität in P2P Netzwerken?
	\li Wie werden Beim Tapistry Netzwerk neue Daten eingefügt und gelöscht? Was passiert dabei? Wie wird ein Objekt lokalisiert?
}
\se{DHT}
\olp{
	\li Was ist eine verteilte Hashtable und was kann sie?
	\li Wie funktioniert ein Basic Lookup? Nachteile?
	\li Was ist eine ``Finger Table''? Wie funktioniert sie? Wie funktionieren Successor Lists?
	\li Was passiert, wenn ein neuer Node eingefügt wird?
}


\ch{Zeit und globale Zustände}
\se{Zeit}
\olp{
	\li Was ist Zeit? Welche Bedeutung hat Zeit in einem Informatiksystem? Was passiert, wenn unterschiedliche Applikationen/Dienste unterschiedliche Zeiten haben?
	\li Warum sind Transaktionen auf einen konsistente Zeittaktung angewiesen?
	\li Wie ist der Zustand eines Prozesses definiert?
	\li Abb. \ref{proNaAt}: Erklären Sie anhand der Abbildung, was Ereignisse und Messages sind und welchen Einfluss die Laufzeit der Meldungen zwischen den Prozessen auf den Betrieb hat.
		\img{img/v9.1.jpg}{}{0.75}{proNaAt}
	\li Erklären Sie den Unterschied zwischen physischer und logischer Uhr.
	\li Warum darf beim Synchronisieren von Zeit niemals eine Komponente rückwärtskorrigiert werden?
	\li Erklären Sie das Konzept der logischen Uhr. Welche Eigenschaften hat die logische Uhr.
	\li Warum ist in verteilten Systemen eine exakte Zeitsynchronisation nicht möglich?
	\li Erklären Sie den Lamport Algorithmus.
	\li Machen Sie ein Beispiel mit dem Lamport Algorithmus mit vier unabhänigen Prozessen.
	\li Welches grundsätzliche Problem besitzt die Lamport Zeit?
	\li Erklären Sie die Vektorzeit. Inwiefern löst sie das Problem der Lamport Zeit?
	\li Wie ermitteln Sie bei der Vektorzeit die effektive Zeit bei einem Prozess? Erklären Sie genau: Was passiert bei der Lamportzeit und bei der Vektorzeit jeweils bei den folgenden Ereignissen: Initialisierung, lokales Ereignis, Sendeereignis, Empfangsereignis.
	\li Warum ist die Vektorzeit ireflexiv, asymetrisch und transitiv?
	\li Liefert die Vektorzeit mehr Informationen über zeitliche Zusammenhänge als die Lamport Zeit?
}

\se{globale Zustände}
\olp{
	\li Nennen Sie einige Anwendungsfälle, bei denen ein globaler Zustand entscheidend ist.
	\li Wie ist die History eines Prozesses definiert.
	\li Was ist ein konsistenter Schnitt?
	\li Erklären Sie den Snapshotalgorithmus von Chandy und Lamport.
	\li Welche Bedingungen muss ein Kanal erfüllen?
	\li Warum ist der globale konsistente Zustand mit keinem lokalen Zustand identisch, aber trotzdem konsistent?
}



\ch{Verteilte Synchronisation}
\se{Verteilten Ausschluss}
\olp{
	\li Erklären Sie, wie verteilte Synchronisation funktioniert.
	\li Welche drei Anforderungen werden an verteilten wechselseitigen Ausschluss gestellt?
	\li Skizzieren Sie, wie ein verteiltes wechselseitiges Ausschlusssystem mit zentralem Server funktioniert.
		\oli{
			\li Was passiert, wenn der zentrale Server ausfällt?
			\li Welche Probleme können auftreten?
			\li Was passiert, wenn das Token verloren geht?
		}
	\li Skizzieren Sie, wie ein verteiltes wechselseitiges Ausschlusssystem mit Tokenring funktioniert.
		\oli{
			\li Was passiert, wenn ein Knoten, insbesondere der aktuelle Master, ausfällt?
			\li Was passiert, wenn das Token verloren geht?
			\li Was können für Probleme auftreten?
			\li Welchen Nachteil hat der Tokenring bzgl. Bandbreitenbedarf? Warum?
		}
	\li Erklären Sie, wie der Ricart und Agrawala’s Algorithmus funktioniert. Wo liegt der Schwachpunkt?
	\li Erklären Sie den Maekawa’s Algorithmus.
}

\se{Wahlen}
\olp{
	\li Wie funktioniert eine Ring basiert Wahl?
	\li Wie viele Nachrichten werden im Worst Case versandt?
	\li Wie funktioniert der Bully Algorithmus?
}


\se{Multicast Kommunikation}
\olp{
	\li Welches grundsäzliche Problem ergibt sich mit Multicast, insbesondere wenn man z.B. einen Ausschlussalgorithmus umsetzt, der von allen Teilnehmern eine Bestätigung erwartet, dass sie sich nicht im kritischen Bereich befinden?
	\li Erklären Sie den ISIS Algorithmus
}


\se{Konsens}
\olp{
	\li Welches Problem haben Sie, mit drei Prozessen einen Konsens zu finden?
	\li Erlären Sie das System der Byzantinischen Generäle.

}


\ch{Modellierung verteilter Systeme}
\olp{
	\li Erklären Sie den Unterschied zwischen Parallelität und Nebenläufigkeit.
	\li Skizzieren Sie ein Producer/Consumer Szenario. Beschreiben Sie das erstellte Modell formal.
	\li Was sind Vor- und Nachbereich? Was ist Konzession?
	\li Was sind Markierungen im Petri Netz?
	\li Skizzieren Sie den Erreichbarkeitsgraph für das oben beschrieben Producer/Consumer Szenario.
	\li Skizzieren Sie, wie Sie mit einem Petrinetz wechselseitigen Ausschluss realisieren.
	\li Erklären Sie den Crosstalkalgorithmus.
	\li Was sind reaktive Systeme?
	\li Was ist das ``Labelled Transition System''?
}

\se{CSP}
\olp{
	\li Was ist CPS?
	\li Wo ist der Unterschied zwischen Skip und Stop?
	\li Erklären Sie die Präfix Notation.
	\li Was sind Präzedenzen?
	\li Wie realisieren Sie Rekursion?
	\li Wie realisieren Sie einen choice oder einen switch?
	\li Was sind Traces? Wie definieren Sie Operationen auf Traces?
	\li Wie definieren Sie Kommunikationsevents?
	\li Was ist eine interne (nichtdeterministische) Auswahl?
}



















\end{document}
