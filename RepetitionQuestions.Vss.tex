%Pakete;
%A4, Report, 12pt
\documentclass[ngerman,a4paper,12pt]{scrreprt}
\usepackage[a4paper, right=20mm, left=20mm,top=20mm, bottom=30mm, marginparsep=5mm, marginparwidth=5mm, headheight=7mm, headsep=15mm,footskip=15mm]{geometry}

%Papierausrichtungen
\usepackage{pdflscape}
\usepackage{lscape}

%Deutsche Umlaute, Schriftart, Deutsche Bezeichnungen
\usepackage[utf8]{inputenc}
\usepackage[T1]{fontenc}
\usepackage[ngerman]{babel}

%quellcode
\usepackage{listings}

%tabellen
\usepackage{tabularx}

%listen und aufzählungen
\usepackage{paralist}

%farben
\usepackage[svgnames,table,hyperref]{xcolor}

%symbole
\usepackage{latexsym,textcomp}

%font
\usepackage{helvet}
\renewcommand{\familydefault}{\sfdefault}

%Abkürzungsverzeichnisse
\usepackage[printonlyused]{acronym}

%Bilder
\usepackage{graphicx} %Bilder
\usepackage{float}	  %"Floating" Objects, Bilder, Tabellen...
\usepackage[space]{grffile} %Leerzechen Problem bei includegraphics
\usepackage{wallpaper} %Seitenhintergrund setzen
\usepackage{transparent} %Transparenz

%for
\usepackage{forloop}
\usepackage{ifthen}

%Dokumenteigenschaften
\title{Repetitionsfragen Vss}
\author{Tobias Blaser}
\date{\today{}, Rapperswil}


%Kopf- /Fusszeile
\usepackage{fancyhdr}
\usepackage{lastpage}

\pagestyle{fancy}
	\fancyhf{} %alle Kopf- und Fußzeilenfelder bereinigen
	\renewcommand{\headrulewidth}{0pt} %obere Trennlinie
	\fancyfoot[L]{Seite \thepage/\pageref{LastPage}} %Fusszeile mitte
	\fancyfoot[R]{\today{}} %Fusszeile rechts
	\renewcommand{\footrulewidth}{0.4pt} %untere Trennlinie

%Kopf-/ Fusszeile auf chapter page
\fancypagestyle{plain} {
	\fancyhf{} %alle Kopf- und Fußzeilenfelder bereinigen
	\renewcommand{\headrulewidth}{0pt} %obere Trennlinie
	\fancyfoot[L]{Seite \thepage/\pageref{LastPage}} %Fusszeile mitte
	\fancyfoot[R]{\today{}} %Fusszeile rechts
	\renewcommand{\footrulewidth}{0.4pt} %untere Trennlinie
}

\usepackage{changepage}

% Abkürzungen für Kapitel, Titel und Listen
\input{commands/shortcutsListAndChapter}
\input{commands/TextStructuringBoxes}

%links, verlinktes Inhaltsverzeichnis, PDF Inhaltsverzeichnis
\usepackage[bookmarks=true,
bookmarksopen=true,
bookmarksnumbered=true,
breaklinks=true,
colorlinks=true,
linkcolor=black,
anchorcolor=black,
citecolor=black,
filecolor=black,
menucolor=black,
pagecolor=black,
urlcolor=black
]{hyperref} % Paket muss unbedingt als letzes eingebunden werden!

\usepackage{graphicx}
\begin{document}

% Inhaltsverzeichnis
\tableofcontents
\clearpage

\ch{Systemmodelle}
\ol
	\li Was ist ein Architekturmodell?
	\li Erklären Sie die folgenden Systemarchitekturen:
		\ul
			\li Client/Server Modell
			\li Proxy-Server und Cache
			\li Peer-to-Peer
			\li Applets
			\li Thin Client / Fat Server
			\li MVC Pattern
			\li Three Tier
			\li ISO OSI Layer Modell
		\ulE
	\li Was sind Ubiquitäre Systeme?
	\li Was ist das Interaktionsmodell?
	\li Was ist das Fehlermodell?
	\li Was ist das Sicherheitsmodell?
	\li Erklären Sie den RM-ODP Architekturansatz.
\olS


\ch{Interprozesskommunikation}
\olR
	\li Stellen Sie synchroner asynchroner Kommunikation gegenüber.
	\li Was sind Sockets und Ports?
	\li Welche Probleme können bei IP, TCP oder UDP auftreten? Wie stark garantiert ist die Übertragung?
	\li Erklären Sie das Fehlermodell von UDP.
	\li Wann setzen Sie UDP ein?
	\li Programmieren Sie einen einfachen UDP Sender und Empfänger. Machen Sie selbiges für TCP.
	\li Wie wirken sich Blockierungen und Timeouts auf die send() und receive() Methoden von UDP in Java aus?
	\li Erklären Sie das Fehlermodell von TCP.
	\li Erklären Sie, was Gruppenkommunikation ist, und wie sie funktioniert. Erklären Sie IP Multicast.
	\li Erklären Sie, externes und internes Marshalling.
	\li Stelle Sie direkter Socketnutzung der Nutzung durch Middleware gegenüber.
	\li Was passiert bei der Objektserialisierung in Java? Wozu können Sie sie nutzen? Wo liegen die Fallstricke? Welche Nachteile hat sie?
	\li Bauen Sie ein kleines Beispielprogramm, in dem Sie ein Objekt Serialisieren, über einen UDP Socket versenden, empfangen und das Objekt wiederherstellen und Attribute ausgeben.
	\li Wie markieren Sie nicht zu serialisierende Objekte? Wie können Sie die Serialisierungsprozedur anpassen?
	\li Wozu dient die SerialVerUID? Wo kann es damit Probleme geben?
	\li Was können Sie mit Java2XML machen?
	\li Was müssen Sie bei der Referenzierung von remote Objekten in einem Verteilten System beachten?
	\li Wozu dienen die NIO-Channels in Java?
\olS


\ch{Verteilte Objekte und entfernte Aufrufe}
\olR
	\li Warum sollte der Benutzer erkennen können, ob es sich bei einem Methodenaufruf um einen lokalen oder entfernten Aufruf handelt?
	\li Welche Dienste sollte die Middleware zur Verfügung stellen?
	\li Erklären Sie das RRA Protokoll schematisch. Welche Probleme führt es mit sich?
	\li Zeigen Sie das Fehlermodell der RRA Protokolls auf. (Server und Clientseitig)
	\li Erklären Se die vier Arten von Aufrufsemantik.
	\li Welche Idee steckt hinter 'Remote Procedure call'? Welche Aufrufarten machen Sinn, welche nicht? (Call by Value, Call by Reference, Zugriff auf lokale Variablen des entfernten Systems)?
	\li Was ist eine IDL? Wie ist die CROBA IDL aufgebaut?
	\li Was ist RPC und wie funktioniert es?
	\li Was ist der Unterschied zwischen einer Methode und einer Procedure?
	\li Was ist RMI? Wie funktioniert es?
\olS

\se{RMI}
\olR
	\li Skizzieren SIe konzepthaft, was sie alles unternehmen müssen, um einen remote Methodenaufruf mit RMI realisieren zu können.
	\li Abb. \ref{rmiprox}: Erklären Sie die Grafik.
		\img{img/v4.3.jpg}{}{0.75}{rmiprox}
	\li Welche Rolle übernehmen der Proxy und welche der Skeletton?
	\li Welche Aufgabe übernimmt der Dispatcher?
	\li Wie realisieren Sie bei RMI eine Objektübergabe by Value? Wie können Sie Ansatzweise 'Object by Reference' realisieren?
	\li Erklären Sie, was remote Garbage Collection ist.
	\li Wozu dient das dynamische Laden von Klassen? Wie realisieren Sie es?
	\li Erklären Sie, wie sie einen RMI Server schlafen legen können und nur dann aufwecken, wenn wirklich etwas ansteht.
\olS




\end{document}
